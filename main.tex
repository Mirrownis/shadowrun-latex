\documentclass[9pt, final]{extarticle}

% enable/disable either print or fancy layout
\usepackage{_fancy}
%\usepackage{_print}

% Individuelle Dokumenteinstellungen
\newcommand{\Name}{Kabelsalat} % Dokumentname
\newcommand{\Info}{//Dateiupload von Unknown} % Dokumentinfo
\newcommand{\Number}{\#003} % Dokumentnummer
\newcommand{\Category}{Die Matrix} % Dokumentnummer
\begin{document}
\begin{multicols}{2}
% % % % % % % % % %
% Dokument-Körper %
% % % % % % % % % %

Ihr habt euch bestimmt schon einmal gefragt, woher die coolen Shirts kommen, die man in keinem Shop der  Matrix findet. Man könnte natürlich davon ausgehen, dass es 
\section{10 Bands, neu im Sortiment}
\textbf{1 Wesley:} Diese Feathercore-Band ist dafür bekannt, bekannte Kinderlieder aus aller Welt neu zu interpretieren und aufzusetzen. Sie sind besonders bei gestressten Konzerndrohnen wegen der samtweichen Stimme des Frontmanns Seth Rees beliebt.\\
\textbf{2 Angels Anonymous:} Eine Radpunk-Truppe aus Frankreich, die vor allem  das Ende der Welt besingen (natürlich durch nuklearen Winter) und bei ihren Performances alle Register ziehen. Bei den meisten Konzerten treten sie mit ihrer eigenen Pfeifenorgel und ihrem eigenen Chor auf.\\
\textbf{3 Go!Mammut:} Trotz der leidenschaftlichen Abneigung seitens aller Kritiker hat sich diese Glam-Rock Band international durchsetzen können. Mit aufwendigen Kostümen und spektakulären Bühnenshows ziehen sie weitaus größere Fangemeinden an, als ihre gemeinhin als mittelmäßig eingestufte Musik vermuten lassen würde.\\
\textbf{4 Full Payload:} Diese komplett vom Label Regency Megamedia ausgehobene und zusammengestellte Bugstomp-Band ist die Überraschung des Jahres, denn obwohl sie hinter den Kulissen wie eine klassische Boyband organisiert und gehandelt werden, ist das martialische Image der Mitglieder ungebrochen. Sie haben es bisher noch nicht in die öffentlichen Matrixradios geschafft, aber das scheint nur noch eine Frage der Zeit zu sein.\\
\textbf{5 Or1G1N:} Töne von KIs für KIs: die Matrixentität 'Or1G1N' hat mehr als einmal darauf hingewiesen, dass seine Werke für organische Ohren gar keinen Sinn ergeben können und nur für andere Künstliche Intelligenzen verständlich seien. Dies hält jedoch eine treue Fangemeinde nicht davon ab, jede Sekunde mittels Audioanalysatoren und Mustererkennungsalgorithmus zu analysieren, um vielleicht doch an die geheimen Nachrichten zu gelangen. Bisher erfolglos.\\
\textbf{6 Deathless:} Diese für ihre brutalen Texte, erwachten Bandmitglieder und Instrumente in Waffenform berühmt-berüchtigte Gruppe aus der ADL liegt in einem Graubereich zwischen Wizmetal, Deathmetal und EBM. Wenn man einen gewaltigen Moshpit zum Austoben sucht, ist man auf ihren Konzerten genau richtig.\\
\textbf{7 Overgrowth:} Die Begründer des Primalpunk sind für ihre politischen Texte und ihr ökologisches Engagement bekannt. Durch die enge Verbindung einiger Mitglieder zur englischen Neuen Druidenbewegung ist die Band in Großbritannien verboten, was jedoch ihrer Beliebtheit keinen Abbruch tut.\\
\textbf{8 !-Inc:} Die elfische Rapperin !-Inc begann ihre Karriere wie viele andere als Straßenrapperin, bevor Horizon auf sie aufmerksam wurde und über Nacht zum Star machte. Durch ihre offene pro-Konzern Haltung und konnte sie die Herzen vieler Bürodrohnen für sich gewinnen, bleibt aber bisher trotzdem unabhängig von einem Label.\\
\textbf{9 Bulletjumper:} Das DJ-Duo aus Atzlan hat sich dem Speedcore verschrieben und mischt damit seit einigen Monaten die Clubszene auf. Ihr erster internationaler Hit "Shock and Awe" besteht ausschließlich aus Audioclips von Urban Brawl-Übertragungen und sie haben sich seitdem einen Namen dafür gemacht, aus so gut wie jeder Art von Geräusch einen Track machen zu können.\\
\textbf{10 Radio-Demon:} Vor einigen Monaten betrat dieser rein virtuelle Künstler die Weltbühne und übernahm prompt die Plex-Szene mit einer Welle aus Nostalgie und Bitrock. Es gab bisher keinen Hinweis auf die Identität des Musikers, er oder sie verbreitet aber neue Songs ausschließlich anonym über Sharing-Hosts, sodass es bezeiten einige Tage braucht, bis sie von der Fangemeinde gefunden werden.

\begin{table_weapon}
NanoSeal & Hard & 12R & 800\textyen & Containment Breacher & Hard  & 16F &	2.800\textyen &
Trojan Horse\nl
NanoSeal & Hard & 12R & 800\textyen & Containment Breacher & Hard  & 16F &	2.800\textyen &
Trojan Horse\nl
\end{table_weapon}

\end{multicols}
\end{document}